%______________________________________________________________________________________________________________________
% @brief    LaTeX2e Resume for Amanda M. Bachmann

\documentclass[margin,line]{resume}
\usepackage{bibentry}
\makeatletter\let\saved@bibitem\@bibitem\makeatother
\usepackage[breaklinks]{hyperref}
\def\UrlBreaks{\do\/\do-}
\makeatletter\let\@bibitem\saved@bibitem\makeatother
%\usepackage{doi}
\hypersetup{
    colorlinks=true,%
    linkcolor=blue,%
    filecolor=magenta,%
    urlcolor=cyan,%
    breaklinks=true
}
\urlstyle{same}
\usepackage{xspace}
%\textheight=10.75in
\newcommand{\Cyclus}{\textsc{Cyclus}\xspace}%

% Trying to bold my name in the bib
\usepackage{xstring}
\def\FormatName#1{%
  \IfSubStr{#1}{Bachmann}{\textbf{#1}}{#1}%
}


%______________________________________________________________________________________________________________________
\begin{document}
\name{\Large Amanda M. Bachmann}
\begin{resume}

\bibliographystyle{abbrvurl-huff}
\nobibliography{cv}

    % Contact Information
    \section{\mysidestyle Contact\\Information}
    NEUP Fellow, Graduate Research Assistant \hfill mobile: (813) 495-5698 \vspace{0mm}\\\vspace{0mm}%
    \textsl{University of Illinois, Urbana-Champaign} \hfill e-mail: amandab7@illinois.edu            \vspace{0mm}\\\vspace{0mm}%
        \textsl{Nuclear, Plasma, and Radiological Engineering}
    \vspace{-2mm}\\\vspace{-6mm}%

    %__________________________________________________________________________________________________________________
    % Resume Objective
    %\section{\mysidestyle Objective}
                %Seeking research and teaching opportunities in nuclear engineering and scientific computation.%
    %__________________________________________________________________________________________________________________
    % Research Interests
    %\section{\mysidestyle Research\\Interests}
    %            Advanced nuclear reactors and fuel cycles, nuclear non-proliferation,
    %            nuclear fuel cycle analysis,
    %            scientific computation.
    %\vspace{-2mm}
    %__________________________________________________________________________________________________________________
    % Academic Appointments
    %           \vspace{-2mm}
    %\section{\mysidestyle Academic\\Appointments}
    %           \vspace{-2mm}\\\vspace{-3mm}%
    %__________________________________________________________________________________________________________________
    % Education
    \section{\mysidestyle PhD}
    \textbf{University of Illinois Urbana-Champaign}, \textsc{Nuclear Engineering}\hfill \textbf{ Aug 2020 -- Present}\vspace{-3mm}\\\vspace{-1mm}%
    \begin{list2}
        \item Nuclear Fuel Cycle Transition to High Assay Low Enriched Uranium Fueled Reactors
        \item GPA: 3.94/4.0
    \end{list2}\vspace{-4mm}
    %GPA: /4.04,0
    \section{\mysidestyle MS}
    \textbf{University of Tennessee, Knoxville}, \textsc{Nuclear Engineering}\hfill\textbf{Aug 2019 -- Aug 2020}\vspace{-3mm}\\\vspace{-1mm}%
    \begin{list2}
        \item Empirical Modeling of Used Nuclear Fuel Signatures Radiation Emissions for Safeguards Purposes
        \item GPA: 3.96/4.0
    \end{list2}\vspace{-4mm}
    %GPA: 3.96/4.0
    \section{\mysidestyle BS}
    \textbf{University of Tennessee, Knoxville}, \textsc{Nuclear Engineering}\hfill\textbf{Aug 2015 -- May 2019}\vspace{-3mm}\\\vspace{-1mm}%
    \begin{list2}
        \item Minors: Material Science \& Engineering; Nuclear Decommissioning \& 
              Environmental Management
        \item GPA: 3.91/4.0
    \end{list2}\vspace{-2mm}
    
    %__________________________________________________________________________________________________________________
    % Research Experience
    \section{\mysidestyle Research\\Experience}
    \textbf{Argonne National Lab}, Lemont, IL\hfill\textbf{May 2022 --Present}\\
    \vspace{-5mm}
                \textsl{Visiting Student, Reactor and Fuel Cycle Analysis Group}\\
        
    \textbf{University of Illinois at Urbana-Champaign}, Urbana, IL \hfill \textbf{Aug 2020 -- Present} \\
    \vspace{-5mm}
        \textsl{NEUP Fellow, Advanced Reactors and Fuel Cycles Group} \\
        
    \textbf{University of Tennessee, Knoxville}, Knoxville, TN \hfill \textbf{Aug 2019 -- Aug 2020}\\
    \vspace{-5mm}
        \textsl{Graduate Research Assistant, Coble Research Lab} \\ 
    
    \textbf{Oak Ridge National Laboratory}, Oak Ridge, TN \hfill \textbf{May 2019 -- Aug 2019}\\
    \vspace{-5mm}        
        \textsl{NESLS Intern, Radiation Transport High Performance Computing Methods and Applications Team}\\               
    
    \textbf{University of Tennessee, Knoxville}, Knoxville, TN\hfill\textbf{Oct 2015 -- May 2019}\\
        \vspace{-5mm}            
        \textsl{Undergraduate Research Assistant, Coble Research Lab}\\
        %\vspace{-5mm}

    %__________________________________________________________________________________________________________________
    
    %__________________________________________________________________________________________________________________

    %__________________________________________________________________________________________________________________
    %            
    \section{\mysidestyle Journal\\Publications}
    %\section{\mysidestyle Refereed\\Journal\\Publications} % also, vspace below
      \begin{bibenum}
      \item \bibentry{bachmann_enrichment_2021} 
            %\vspace{-3mm}
      \item \bibentry{bachmann_comparison_2021}
            %\vspace{-2mm}
      \end{bibenum}
      %\vspace{2mm} % add this back if you return to Refereed Journal..
      %\section{\mysidestyle Submitted}
      %\begin{bibenum}
      %\end{bibenum}
    %\section{\mysidestyle Refereed\\Conference\\Proceedings}

    %__________________________________________________________________________________________________________________
    % Submitted Journal Publications
    %           \vspace{-2mm}
    %\section{\mysidestyle Submitted\\Journal\\Publications}
    %            Submitted Journal Publications \vspace{-2mm}\\\vspace{-3mm}%
    %__________________________________________________________________________________________________________________
    % Conference Publications

    %__________________________________________________________________________________________________________________
    % Honors and Awards
    \section{\mysidestyle Select Honors \\and Awards}
    Presidential Citation, American Nuclear Society \hfill \textbf{2022}\vspace{.5mm}\\
    ANS Student Sections Commendations, ANS Student Sections Committee \hfill \textbf{2020}\vspace{.5mm}\\%
    U.S. WIN Region II Leadership Award, U.S. WIN Region II \hfill \textbf{2019}\vspace{.5mm}\\%
    Girl Scout Gold Award, Girl Scouts of America \hfill \textbf{2015}\vspace{.5mm}\\%
    \vspace{-7mm}
    %__________________________________________________________________________________________________________________
    % Technical Reports
    %__________________________________________________________________________________________________________________
    % Other Publications
    %__________________________________________________________________________________________________________________
    % Media Coverage
    %__________________________________________________________________________________________________________________
    % Invited Talks

    %__________________________________________________________________________________________________________________
    % Panels
   %__________________________________________________________________________________________________________________
        %__________________________________________________________________________________________________________________
    % Advising
        %__________________________________________________________________________________________________________________
    % Computer Skills
    \section{\mysidestyle Scientific\\Computing\\Skills}
                \textbf{Languages} \hfill Python\vspace{.5mm}\\%
                %\textbf{Build Systems} \hfill make, CMake, automake\vspace{.5mm}\\%
                \textbf{Databases} \hfill HDF5\vspace{.5mm}\\%
                \textbf{Test Frameworks} \hfill nose\vspace{.5mm}\\%
                \textbf{Version Control} \hfill git\vspace{.5mm}\\%
                \textbf{Other Tools} \hfill \LaTeX, MatLab, \Cyclus, MCNP, ORIGAMI, ORIGEN\vspace{.5mm}%
        \vspace{-3mm}
    %__________________________________________________________________________________________________________________
    % Professional, Student Organizations

    \section{\mysidestyle Select\\Professional\\Service}
                \textbf{Co-Vice Chair}, Nuclear Engineering Student Delegation \hfill\textbf{2022}\\
                \textbf{Student Director}, Board of Directors, ANS \hfill \textbf{2021--Present}\vspace{.5mm}\\%
                \textbf{DEI Co-Chair}, 2022 ANS Student Conference Planning Committee \hfill \textbf{2021--2022}\vspace{.5mm}\\%
                \textbf{Co-Vice Chair}, Student Sections Committee, ANS \hfill \textbf{2019-2021}\\
                \textbf{Twitter Lead}, Communications Committee, U.S. WIN  \hfill \textbf{2018--Present}\vspace{.5mm}\\%
                \textbf{Member}, Region II Conference Planning Committee, U.S. WIN  \hfill \textbf{2018--2019}\vspace{.5mm}\\%


    %__________________________________________________________________________________________________________________
    % References

%\section{\mysidestyle References}
%\texsl{Available upon request}

%______________________________________________________________________________________________________________________


    %__________________________________________________________________________________________________________________
\end{resume}

\end{document}


%______________________________________________________________________________________________________________________
% EOF

